\begin{course}{Дискретная математика~1}{dm1}

\hours{I курс, сентябрь--декабрь.}{5~зачётных единиц, 2~часа лекций, 2~часа практик в~неделю.}{зачёт, экзамен.}

\myabstract{
Одна из~основных целей данного курса~--- научиться строить 
и~строго анализировать математические модели, возникающие 
в~программировании и~компьютерных науках. 
Много внимания будет уделено формальным 
доказательствам корректности: разберёмся, какие бывают методы
 доказательства; узнаем, как доказывать существование, оптимальность 
и~универсальность; увидим много примеров того, как интуитивно 
кажущееся правильным доказательство оказывается формально
неверным (что на~практике может привести к~сбою оборудования
или программного обеспечения). Также узнаем, как эффективно 
перебирать различные комбинаторные объекты и вычислять их количество,
не перечисляя их явно. Увидим, как часто это помогает при построении 
эффективных алгоритмов. Наконец, начнём изучать теорию 
вероятностей~--- раздел математики, возникающий в~самых разных 
областях компьютерных наук: в~алгоритмах, теории игр, криптографии,
обработке сигналов и~других. Узнаем, как формально определять
и~оценивать вероятности различных случайных процессов. Научимся 
пользоваться стандартными методами оценки разных параметров 
случайных величин~--- среднего, дисперсии, уклонений.

Курс содержит большое количество задач разной степени сложности~--- 
как на~доказательство, так и~на программирование. 
Задачи на доказательство призваны дать слушателям возможность
потренироваться кратко и~строго записывать рассуждения. Задачи
на~программирование покажут, как именно возникающие дискретные
структуры используются на практике, и~помогут разобраться во~всех 
деталях этих структур.
};

%\beforeandafter{
\rightblock{
  \item Математика: школьная программа 
  (доказательства, основные функции: логарифм, многочлен, экспонента).
  \item Программирование: базовое владение языком 
  программирования \texttt{Python} (ввод-вывод, циклы, рекурсия).
}{
  \item Во всех последующих математических дисциплинах.
  \item Для построения и~анализа алгоритмов и~эффективных программ.
  \item Для понимания курсов:
    \hyperref[ads1]{Алгоритмы и~структуры данных}, 
    \hyperref[prob]{Теория вероятностей},
    \hyperref[fl]{Теоретическая информатика},
    \hyperref[stat]{Математическая статистика}, 
    \hyperref[ml1]{Машинное обучение}.
}{
%\syllabus{
  \item Доказательства: существование, оптимальность, универсальность.
  \item Множества: мощности, диагональный аргумент, порядки, 
    цепи/антицепи.
  \item Перебор и~подсчёт: подмножества, перестановки, размещения,
    сочетания, скобочные последовательности, разбиения, 
    рекуррентные соотношения.
  \item Вероятность: события, 
    условная вероятность и~независимость,
    случайные величины, уклонения, вероятностный метод.
}{
%\outcomes{
  \item строго и~коротко записывать доказательства;
  \item оценивать время работы алгоритмов и~программ;
  \item оценивать вероятности;
  \item применять идеи дискретной математики в~разных областях IT;
  \item реализовывать перебор, кодирование и~шифрование, 
    сведения к~задаче выполнимости.
}
\end{course}
